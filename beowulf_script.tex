\documentclass[a4paper]{article}

% Table of contents depth (currently unused)
\setcounter{tocdepth}{3}

% Section numbering depth (zero for no numbering)
\setcounter{secnumdepth}{0}

% latex package inclusions here
%\usepackage{fullpage}
\usepackage{hyperref}
\usepackage{tabulary}
%\usepackage{amsthm}

% set up BNF generator
%\usepackage{syntax}
%\setlength{\grammarparsep}{10pt plus 1pt minus 1pt}
%\setlength{\grammarindent}{10em} 

% set up source code inclusion
\usepackage{listings}
\lstset{
  tabsize=2,
  basicstyle = \ttfamily\small,
  columns=fullflexible
}
% Usage for the above like so:
% \begin{lstlisting}
%   CODE CODE CODE
% \end{lstlisting}

% in-line code styling (same style as listing)
\newcommand{\shell}[1]{\lstinline{#1}}

% Line and paragraph spacing
\newenvironment{linewise}
  {\parindent=0pt
   \obeyspaces\obeylines
   \begingroup\lccode`~=`\^^M
   \lowercase{\endgroup\def~}{\par\leavevmode}}
  {\ignorespacesafterend}

%%%%%%%%%%%%%%%%%%%%%%%%%%%%%%%%%%%%%%%%%%%%%%%%%%%%%%%%%%%%%%%%%%%%%%%%%%%%%%%

\begin{document}
\title{Beowulf Script}
\date{2016}
\author{
Daniel Clay \\ 
}
\maketitle

%%%%%%%%%%%%%%%%%%%%%%%%%%%%%%%%%%%%%%%%%%%%%%%%%%%%%%%%%%%%%%%%%%%%%%%%%%%%%%%
\section{Introduction}
%%%%%%%%%%%%%%%%%%%%%%%%%%%%%%%%%%%%%%%%%%%%%%%%%%%%%%%%%%%%%%%%%%%%%%%%%%%%%%%

\subsection{Cast}%%%%%%%%%%%%%%%%%%%%%%%%%%%%%%%%%%%%%%%%%%%%%%%%%%%%%%%%

In order of appearance:
\begin{description}
    \item[The Bard] Narrator
    \item[Hrothgar] King of the Spear-Danes (Scyldings)*
    \item[Grendel] An evil monster, descendant of Cain*
    \item[Beowulf] A hero of the Geatish people
    \item[Beowulf's companions] 14 Geatish thegns*
    \item[Wulfgar] Hrothgar's Counsellor*
    \item[Unferth] Hrothgar's spokesman*
    \item[Wealhtheow] Queen of the Spear-Danes*
    \item[Hrethic \& Hrothmund] Sons of Hrothgar*
    \item[Grendel's Mother] A monstrous ogress*
    \item[Ashhere] Hrothgar's Counsellor*
    \item[Hygelac] King of the Geats**
    \item[Hygd] Queen of the Geats**
    \item[The Dragon] A dragon, awakened from it's slumber**
    \item[A Slave] Who awakens the dragon**
    \item[Wiglaf] Beowulf's faithful companion**
    \item[Beowulf's companions] 7(?) Geatish thegns**
\end{description}

Of the cast, most of Beowulf's companions can be omitted as required, and those
present at the court of Hrothgar (marked with a  *) can be doubled up with those present at the court
of Hygelac (marked with a **; characters who cannot be doubled up are unmarked). Hrethic \& Hrothmund may also be omitted.

Likewise, additional characters may be present at both courts - councillors, warriors and servants.

Most of the minor roles can be played by men or women, as can Grendel, Grendel's
Mother, the Dragon (possibly a two person costume) and the Bard.

In total, the play can be performed by anywhere upwards of 10 people, with an ideal number being around 14-20.

\subsection{Staging}%%%%%%%%%%%%%%%%%%%%%%%%%%%%%%%%%%%%%%%%%%%%%%%%%%%%%%%%%%%

This play works best when some of the character of the mead-hall, for which the
original work was composed, is retained. Staging should be minimal and as far as
possible abstracted away. No scene changes should be required, but can be suggested 
by lighting or simple changes of set.

Lighting by open flame is preferred to lighting by electric light.

\subsection{Structure of the play}%%%%%%%%%%%%%%%%%%%%%%%%%%%%%%%%%%%%%%%%%%%%%

Despite the length of the original poem, which runs to over 3000 lines, the play
should be able to be performed straight through, without an interval.

If an interval is desired, it should happen after Act II.

The poem itself is loosely organised in three acts, characterised by the three 
combats Beowulf takes part in; that with Grendel, with Grendel's Mother, and with the Dragon.

\newpage

%%%%%%%%%%%%%%%%%%%%%%%%%%%%%%%%%%%%%%%%%%%%%%%%%%%%%%%%%%%%%%%%%%%%%%%%%%%%%%%
\section{Script}
%%%%%%%%%%%%%%%%%%%%%%%%%%%%%%%%%%%%%%%%%%%%%%%%%%%%%%%%%%%%%%%%%%%%%%%%%%%%%%%

\subsection{Act 1}%%%%%%%%%%%%%%%%%%%%%%%%%%%%%%%%%%%%%%%%%%%%%%%%%%%%%%%%%%%%%

{\linewise

\centerline{\textbf{Scene 1}}
\centerline{\textit{A mead hall, somewhere in 6th Century Denmark.}}
\centerline{\textit{\textbf{The Bard} addresses the audience as if they were guests in the mead hall.}}

\textbf{The Bard} Listen!
We have heard of the thriving of the throne of Denmark,
how the folk-kings flourished in former days,
how those royal athelings earned that glory.

Was it not Scyld Shefing that shook the halls,
took mead-benches, taught encroaching
foes to fear him – who, found in childhood,
lacked clothing? Yet he lived and prospered,
grew in strength and stature under the heavens
until the clans settled in the sea-coasts neighbouring
over the whale-road all must obey him
and give tribute. He was a good king!

A boy child was afterwards born to Scyld,
a young child in hall-yard, a hope for the people,
sent them by God
Through the northern lands the name of Beow,
the son of Scyld, sprang widely.

At the hour shaped for him Scyld departed,
the hero crossed into the keeping of his Lord.
Then for a long space there lodged in the stronghold
Beowulf the Dane, dear king of his people,
when late was born to him
the lord Healfdene, lifelong the ruler
and war-feared patriarch of the proud Scyldings.

He next fathered four children
that leapt into the world, this leader of armies,
Heorogar and Hrothgar and Halga the Good and Ursula.

Then to Hrothgar was granted glory in battle,
mastery of the field; so friends and kinsmen
gladly obeyed him, and his band increased
to a great company. It came into his mind
that he would command the construction
of a huge mead-hall, a house greater
than men on earth ever had heard of,
and share the gifts God had bestowed on him
upon its floor with folk young and old –
apart from public land and the persons of slaves.

Far and wide (as I heard) the work was given out
in many a tribe over middle earth,
the making of the mead-hall. And, as men reckon,
the day of readiness dawned very soon
for this greatest of houses. Heorot he named it
whose word ruled a wide empire.
He made good his boast, gave out rings,
arm-bands at the banquet. Boldly the hall reared
its arched gables; unkindled the torch-flame
that turned it to ashes. The time was not yet
when the blood-feud should bring out again
sword-hatred in sworn kindred.

It was with pain that the powerful spirit
dwelling in darkness endured that time,
hearing daily the hall filled with loud amusement.
\textit{Grendel} they called this cruel spirit,
the fell and fen his fastness was, the march his haunt.

With the coming of night came Grendel also,
sought the great house and how the Ring-Danes
held their hall when the horn had gone round.
He found in Heorot the force of nobles
slept after supper, sorrow forgotten,
the condition of men. Maddening with rage,
he struck quickly, creature of evil:
grim and greedy, he grasped on their pallets
thirty warriors, and away he was out of there,
thrilled with his catch: he carried off homeward
his glut of slaughter, sought his own halls.

As the day broke, with the dawn’s light
Grendel’s outrage was openly to be seen:
night’s table-laughter turned to morning’s
lamentation. Lord Hrothgar
sat silent then, the strong man mourned,
glorious king, he grieved for his thegns
as they read the traces of a terrible foe,
a cursed fiend. That was too cruel a feud,
too long, too hard!

Nor did he let them rest
but the next night brought new horrors,
did more murder, manslaughter and outrage
and shrank not from it: he was too set on these things.

So Grendel became ruler; against right he fought,
one against all. Empty then stood
the best of houses, and for no brief space;
for twelve long winters torment sat
on the Friend of the Scyldings, fierce sorrows
and woes of every kind; which was not hidden
from the sons of men, but was made known
in grieving songs, how Grendel warred
long on Hrothgar, the harms he did him
through wretched years of wrong, outrage
and persecution.

A great grief was it for the Guardian of the Scyldings,
crushing to his spirit. The council lords
sat there daily to devise some plan,
what might be best for brave-hearted
Danes to contrive against these terror-raids.

This was heard of at his home by one of Hygelac’s followers,
a good man among the Geats, Grendel’s raidings;
he was for main strength of all men foremost
that trod the earth at that time of day;
build and blood matched.

He bade a seaworthy
wave-cutter be fitted out for him; the warrior king
he would seek, he said, over swan’s riding,
that lord of great name, needing men.

The prince had already picked his men
from the folk’s flower, the fiercest among them
that might be found. With fourteen men
he sought sound-wood; sea-wise Beowulf
led them right down to the land’s edge.

Away she went over a wavy ocean,
boat like a bird, breaking seas,
wind-whetted, white-throated,
till the curved prow had ploughed so far
the sun standing right on the second day –
that they might see land loom on the skyline,
then the shimmer of cliffs, sheer fells behind,
reaching capes.

Then the crossing was at an end;
closed the wake. Weather-Geats
stood on strand, stepped briskly up;
a rope went ashore, ring-mail clashed,
battle-girdings. God they thanked
for the smooth going over the salt trails.

There was stone paving on the path that brought
the war-band on its way. The war-coats shone
and the links of hard hand-locked iron
sang in their harness as they stepped along
in their gear of grim aspect to the hall.

\centerline{\textbf{Scene 2}}
\centerline{\textit{Beowulf \& Companions approach the hall.}}
\centerline{\textit{Loud knocking on the door, which then swings open}}
\centerline{\textit{They enter. Unferth and Wulfgar stand and walk over to greet/challenge them.}}

\textbf{Unferth} From whence do you bring these embellished shields, 
grey mail-shirts, masked helmets, 
this stack of spears? I am spokesman here,
herald to Hrothgar. I have not seen
a body of strangers bear themselves more proudly.

It is not exile but adventure, I think,
boldness of spirit, that brings you to Hrothgar.

\textbf{Beowulf} At Hygelac's table we are sharers in the banquet.
Beowulf is my name.
I shall gladly set out to the son of Healfdene,
most famous of kings, the cause of my journey,
lay it before your lord, if he will allow us kindly,
to greet in person his most gracious self.

\textbf{Wulfgar} The Master of the Danes,
Lord of the Scyldings, shall learn of your request.
I shall gladly ask my honoured chief,
giver of arm-bands, about your undertaking,
and soon bear the answer back again to you,
that my gracious lord shall think to make.

\centerline{\textit{Unferth and Wulfgar walk over to Hrothgar to tell him who the strangers are.}}

\textbf{Wulfgar} Men have come here from the country of the Geats,
borne from afar over tha back of the sea;
these battle-companions call the man who leads them Beowulf.

The boon they ask is, my lord, that they may converse with you. 
Do not, kind Hrothgar, refuse them audience in the answer you vouchsafe.
Their war-gear would clearly bespeak them of Earl's rank. 
Indeed, the leader who guided them here seems of great account.

\textbf{Hrothgar} I knew him when he was a child!
It was to his old father, Edgetheow, that
Hrethel the Geat gave in marriage
his one daughter. Well does the son
now pay this call on a proven ally!

 The seafarers used to say, I remember,
who took our gifts to the Geat people
in token of friendship – that this fighting man
in his hand’s grasp had the strength
of thirty other men. I am thinking that
the Holy God, as a grace to us
Danes in the West, has directed him here
against Grendel’s oppression. This good man shall be
offered treasures in return for his courage.

Waste no time now but tell them to come in
that they may see this company seated together.
Make sure to say that they are most welcome
to the people of the Danes.

\centerline{\textit{Unferth and Wulfgar walk back to Beowulf and his companions, who have been waiting silently.}}

\textbf{Wulfgar} The Master of Battles bids me announce,
the Lord of the North Danes, that he knows your ancestry;
I am to tell you all, determined venturers
over the seas, that you are sure of welcome.
You may go in now in your gear of battle,
set eyes on Hrothgar, helmed as you are.

But battle-shafts and shields of linden wood
may here await your words' outcome.

\centerline{\textit{The Geats hand their weapons to two of their number, who remain to guard them.}}
\centerline{\textit{The rest follow Beowulf towards the king, now unarmed but still in armour.}}

\textbf{Beowulf} Health to Hrothgar! I am Hygelac’s kinsman
and serve in his fellowship. Fame-winning deeds
have come early to my hands. The affair of Grendel
has been made known to me on my native turf.

The sailors speak of this splendid hall,
this most stately building, standing idle
and silent of voices, as soon as the evening light
has hidden below the heaven’s bright edge.

Whereupon it was urged by the ablest men
among our people, men proved in counsel,
that I should seek you out, most sovereign Hrothgar.

These men knew well the weight of my hands.
Had they not seen me come home from fights
where I had bound five Giants?
Had I not crushed on the wave
sea-serpents by night in narrow struggle,
broken the beasts? And shall I not try
a single match with this monster Grendel,
a trial against this troll?

To you I now
put one request, Royal Scylding,
Shield of the South Danes, one sole favour
that you’ll not deny me, dear lord of your people,
now that I have come thus far, Fastness of Warriors;
that I alone may be allowed, with my loyal and determined
crew of companions, to cleanse your hall, Heorot.

As I am informed that this unlovely one
is careless enough to carry no weapon,
so that my lord Hygelac, my leader in war,
may take joy in me, I abjure utterly
the bearing of sword or shielding yellow
board in this battle! With bare hands shall I
grapple with the fiend, fight to the death here,
hater and hated! He who is chosen
shall deliver himself to the Lord’s judgement.

If he can contrive it, we may count upon Grendel
to eat quite fearlessly the flesh of Geats
here in this war-hall; has he not chewed
on the strength of this nation? There will be no need, Sir,
for you to bury my head; he will have me gladly,
if death should take me, though darkened with blood.
He will bear my bloody corpse away, bent on eating it,
make his meal alone, without misgiving,
bespatter his moor-lair. The disposing of my body
need occupy you no further then.

But if the fight should take me, you would forward to Hygelac
this best of battle-shirts, that my breast now wears.
The queen of war-coats, it is the bequest of Hrethel
and from the forge of Wayland. 
Fate will take it's course!

\textbf{Hrothgar} So it is to fight in our defence, my friend Beowulf,
and as an office of kindness that you have come to us here!

It is a sorrow in spirit for me to say to any man
what the hatred of Grendel has brought me to in Heorot, 
what humiliation, what harrowing pain. 
My hall-companions, my war-band, are dwindled; 
Weird has swept them into the power of Grendel.

Yet God could easily check the ravages of this reckless fiend!
They often boasted, when the beer was drunk,
and called out over the ale-cup, my captains in battle,
that they would here await, in this wassailing-place,
with deadliness of iron edges, the onset of Grendel.
When morning brought the bright daylight
this mead-hall was seen all stained with blood:
blood had soaked its shining floor,
it was a house of slaughter. More slender grew my
strength of dear warriors; death took them off …

Yet sit now to the banquet, where you may soon attend,
should the mood so take you, some tale of victory.’

\centerline{\textit{A bench is cleared for the Geats, and they all sit.}}
\centerline{\textit{Mead is brought around and laughter is heard. Talking amongst the cast.}}

\centerline{\textbf{Scene 3}}
\centerline{\textit{After some time, an increasingly irritated Unferth stands up.}}

\textbf{Unferth} Is this the Beowulf of Breca's swimming match, 
who strove against him on the stretched ocean,
when for pride the pair of you proved the seas
and for a trite boast entrusted your lives
to the deep waters, undissuadable
by effort of friend or foe whatsoever
from that swimming on the sea? A sorry contest!

Your arms embraced the ocean’s streams,
you beat the wave-way, wove your hand-movements,
and danced on the Spear-Man. The sea boiled with whelming
waves of winter; in the water’s power
you laboured seven nights: and then you lost your swimming-match,
his might was the greater; morning found him
cast by the sea on the coast of the Battle-Reams.

He made his way back to the marches of the Brondings,
to his father-land, friend to his people,
and to the city-fastness where he had subjects, treasure
and his own stronghold. The son of Beanstan
performed to the letter what he had promised to you.

I see little hope then of a happier outcome
– though in other conflicts elsewhere in the world
you may indeed have prospered – if you propose awaiting
Grendel all night, on his own ground, unarmed.

\textbf{Beowulf} I thank my friend Unferth, who unlocks us this tale
of Breca's bragged exploit; 
the beer lends eloquence to his tongue.
But the truth is as I’ve said:
I had more sea-strength, outstaying Breca’s,
and endured underwater a much worse struggle.
    It was in early manhood that we undertook
with a public boast – both of us still
very young men – to venture our lives
on the open ocean; which we accordingly did.

Hard in our right hands we held each a sword
as we went through the sea, so to keep off
the whales from us. If he whitened the ocean,
no wider appeared the water between us.
He could not away from me; nor would I from him.
Thus stroke for stroke we stitched the ocean
five nights and days, drawn apart then
by cold storm on the cauldron of waters;
under lowering night the northern wind
fell on us in warspite: the waves were rough!

The unfriendliness was then aroused of the fishes of the deep.
Against sea-beasts my body-armour,
hand-linked and hammered, helped me then,
this forge-knit battleshirt bright with gold,
decking my breast. Down to the bottom
I was plucked in rage by this reptile-fish,
pinned in his grip. But I got the chance
to thrust once at the ugly creature
with my weapon’s point: war took off then
the mighty monster; mine was the hand did it.
Then loathsome snouts snickered by me,
swarmed at my throat. I served them out
with my good sword, gave them what they asked for:
those scaly flesh-eaters sat not down
to dine on Beowulf, they browsed not on me
in that picnic they’d designed in the dingles of the sea.
Daylight found them dispersed instead
up along the beaches where my blade had laid them
soundly asleep; since then they have never
troubled the passage of travellers over
that deep water-way. Day in the east grew,
God’s bright beacon, and the billows sank
so that I then could see the headlands,
the windy cliffs. “Weird saves oft
the man undoomed if he undaunted be!” –
and it was my part then to put to the sword
nine sea-monsters, in the severest fight
by night I have heard of under heaven’s vault;
a man more sorely pressed the seas never held.
I came with my life from the compass of my foes,
but tired from the struggle. The tide bore me
away on its currents to the coasts of Norway,
whelms of water.

No whisper has yet reached me
of sword-ambushes survived, nor such scathing perils
in connection with your name! Never has Breca,
nor you Unferth either, in open battle-play
framed such a deed of daring with your
shining swords – small as my action was.
You have killed only kindred, kept your blade
for those closest in blood; you’re a clever man, Unferth,
but you’ll endure hell’s damnation for that.

It speaks for itself, my son of Edgelaf,
that Grendel had never grown such a terror,
this demon had never dealt your lord
such havoc in Heorot, had your heart’s intention
been so grim for battle as you give us to believe.
He’s learnt there’s in fact not the least need
excessively to respect the spite of this people,
the scathing steel-thresh of the Scylding nation.
He spares not a single sprig of your Danes
in extorting his tribute, but treats himself proud,
butchering and dispatching, and expects no resistance
from the spear-wielding Scyldings.

I’ll show him Geatish
strength and stubbornness shortly enough now,
a lesson in war. He who wishes shall go then
blithe to the banquet when the breaking light
of another day shall dawn for men
and the sun shine glorious in the southern sky.

\centerline{\textit{The partying resumes, and Wealhtheow enters with the ceremonial mead-cup.}}
\centerline{\textit{She passes it around, starting with Hrothgar, before coming last to Beowulf.}}

\textbf{Wealhtheow} GREETINGS ETC ETC - must write

\centerline{\textit{Beowulf takes the cup and drinks.}}

\textbf{Beowulf} This was my determination in taking to the ocean,
benched in the ship among my band of fellows,
that I should once and for all accomplish the wishes
of your adopted people, or pass to the slaughter,
viced in my foe’s grip. This vow I shall accomplish,
a deed worthy of an earl; decided otherwise
here in this mead-hall to meet my ending-day!

\centerline{\textit{Wealhtheow retrieves the cup and moves back to sit beside Hrothgar.}}

\textbf{The Bard} Then at last Heorot heard once more
words of courage, the carousing of a people
singing their victories; till the son of Healfdene
desired at length to leave the feast,
be away to his night’s rest; aware of the monster
brooding his attack on the tall-gabled hall
from the time they had seen the sun’s lightness
to the time when darkness drowns everything
and under its shadow-cover shapes do glide
dark beneath the clouds. 
The company came to its feet.

\centerline{\textit{All stand. Hrothgar and Beowulf come together to speak.}}

\textbf{Hrothgar} Never since I took up shield and sword
have I at any instance to any man beside,
thus handed over Heorot, as I here do to you.
Have and hold now the house of the Danes!
Bend your mind and your body to this task
and wake against the foe! There’ll be no want of liberality
if you come out alive from this ordeal of courage.

\centerline{\textit{Hrothgar and and his people leave. The Geats begin to take off their armour.}}
\centerline{\textit{Mattresses are taken out and the Geats lie down on them. Beowulf addresses the audience.}}

\textbf{Beowulf} I fancy my fighting-strength, my performance in combat,
at least as greatly as Grendel does his;
and therefore I shall not cut short his life
with a slashing sword – too simple a business.
He has not the art to answer me in kind,
hew at my shield, shrewd though he be
at his nasty catches. No, we’ll at night play
without any weapons – if unweaponed he dare
to face me in fight. The Father in His wisdom
shall apportion the honours then, the All-holy Lord,
to whichever side shall seem to Him fit.

\centerline{\textit{Darkness}}

\newpage

}
\end{document}
