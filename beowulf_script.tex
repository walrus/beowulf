\documentclass[a4paper]{article}

% Table of contents depth (currently unused)
\setcounter{tocdepth}{3}

% Section numbering depth (zero for no numbering)
\setcounter{secnumdepth}{0}

% latex package inclusions here
%\usepackage{fullpage}
\usepackage{hyperref}
\usepackage{tabulary}
%\usepackage{amsthm}

% set up BNF generator
%\usepackage{syntax}
%\setlength{\grammarparsep}{10pt plus 1pt minus 1pt}
%\setlength{\grammarindent}{10em} 

% set up source code inclusion
\usepackage{listings}
\lstset{
  tabsize=2,
  basicstyle = \ttfamily\small,
  columns=fullflexible
}
% Usage for the above like so:
% \begin{lstlisting}
%   CODE CODE CODE
% \end{lstlisting}

% in-line code styling (same style as listing)
\newcommand{\shell}[1]{\lstinline{#1}}

% Line and paragraph spacing
\newenvironment{linewise}
  {\parindent=0pt
   \obeyspaces\obeylines
   \begingroup\lccode`~=`\^^M
   \lowercase{\endgroup\def~}{\par\leavevmode}}
  {\ignorespacesafterend}

%%%%%%%%%%%%%%%%%%%%%%%%%%%%%%%%%%%%%%%%%%%%%%%%%%%%%%%%%%%%%%%%%%%%%%%%%%%%%%%

\begin{document}
\title{Beowulf Script}
\date{2016}
\author{
Daniel Clay \\ 
}
\maketitle

%%%%%%%%%%%%%%%%%%%%%%%%%%%%%%%%%%%%%%%%%%%%%%%%%%%%%%%%%%%%%%%%%%%%%%%%%%%%%%%
\section{Introduction}
%%%%%%%%%%%%%%%%%%%%%%%%%%%%%%%%%%%%%%%%%%%%%%%%%%%%%%%%%%%%%%%%%%%%%%%%%%%%%%%

\subsection{Characters}%%%%%%%%%%%%%%%%%%%%%%%%%%%%%%%%%%%%%%%%%%%%%%%%%%%%%%%%

In order of appearance:
\begin{description}
    \item[The Bard] Narrator
    \item[Hrothgar] King of the Spear-Danes (Scyldings)
    \item[Grendel] An evil monster, descendant of Cain
    \item[Beowulf] A hero of the Geatish people
    \item[Beowulf's companions] 14 Geatish thegns
    \item[Lookout] A man of the Spear Danes who keeps watch for ships
    \item[Wulfgar] Hrothgar's Counsellor
    \item[Unferth] Hrothgar's spokesman
    \item[Wealhtheow] Queen of the Spear-Danes
    \item[Hrethic \& Hrothmund] Sons of Hrothgar
    \item[Grendel's Mother] A monstrous ogress
    \item[Ashhere] Hrothgar's Counsellor
    \item[Hygelac] King of the Geats
    \item[Hygd] Queen of the Geats
    \item[The Dragon] A dragon
    \item[A Slave] Who awakens the dragon
    \item[Wiglaf] Beowulf's companion
\end{description}

Of the cast, most of Beowulf's companions can be omitted as required, and those
present at the court of Hrothgar can be doubled up with those present at the court
of Hygelac. Hrethic \& Hrothmund may also be omitted.

Most of the minor roles can be played by men or women, as can Grendel, Grendel's
Mother, the Dragon and the Bard

\subsection{Staging}%%%%%%%%%%%%%%%%%%%%%%%%%%%%%%%%%%%%%%%%%%%%%%%%%%%%%%%%%%%

This play works best when some of the character of the mead-hall, for which the
original work was composed, is retained. Staging should be minimal and as far as
possible abstracted away. No set changes should be required.

Lighting by open flame is preferred to lighting by electric light.

\subsection{Structure of the play}%%%%%%%%%%%%%%%%%%%%%%%%%%%%%%%%%%%%%%%%%%%%%

Despite the length of the original poem, which runs to over 3000 lines, the play
should be able to be performed straight through, without an interval.

The poem itself is characterised by the three combats Beowulf partakes in; that
with Grendel, with Grendel's Mother, and with the Dragon. 

%%%%%%%%%%%%%%%%%%%%%%%%%%%%%%%%%%%%%%%%%%%%%%%%%%%%%%%%%%%%%%%%%%%%%%%%%%%%%%%
\section{Script}
%%%%%%%%%%%%%%%%%%%%%%%%%%%%%%%%%%%%%%%%%%%%%%%%%%%%%%%%%%%%%%%%%%%%%%%%%%%%%%%

{\linewise

\centerline{\textbf{Scene 1}}
\centerline{\textit{A mead hall, somewhere in Anglo Saxon England}}

\textbf{The Bard} Listen!
We have heard of the thriving of the throne of Denmark,
how the folk-kings flourished in former days,
how those royal athelings earned that glory.

Was it not Scyld Shefing that shook the halls,
took mead-benches, taught encroaching
foes to fear him – who, found in childhood,
lacked clothing? Yet he lived and prospered,
grew in strength and stature under the heavens
until the clans settled in the sea-coasts neighbouring
over the whale-road all must obey him
and give tribute. He was a good king!

A boy child was afterwards born to Scyld,
a young child in hall-yard, a hope for the people,
sent them by God
Through the northern lands the name of Beow,
the son of Scyld, sprang widely.

At the hour shaped for him Scyld departed,
the hero crossed into the keeping of his Lord.
Then for a long space there lodged in the stronghold
Beowulf the Dane, dear king of his people,
when late was born to him
the lord Healfdene, lifelong the ruler
and war-feared patriarch of the proud Scyldings.

He next fathered four children
that leapt into the world, this leader of armies,
Heorogar and Hrothgar and Halga the Good and Ursula.

Then to Hrothgar was granted glory in battle,
mastery of the field; so friends and kinsmen
gladly obeyed him, and his band increased
to a great company. It came into his mind
that he would command the construction
of a huge mead-hall, a house greater
than men on earth ever had heard of,
and share the gifts God had bestowed on him
upon its floor with folk young and old –
apart from public land and the persons of slaves.

Far and wide (as I heard) the work was given out
in many a tribe over middle earth,
the making of the mead-hall. And, as men reckon,
the day of readiness dawned very soon
for this greatest of houses. Heorot he named it
whose word ruled a wide empire.
He made good his boast, gave out rings,
arm-bands at the banquet. Boldly the hall reared
its arched gables; unkindled the torch-flame
that turned it to ashes. The time was not yet
when the blood-feud should bring out again
sword-hatred in sworn kindred.

It was with pain that the powerful spirit
dwelling in darkness endured that time,
hearing daily the hall filled with loud amusement.
\textit{Grendel} they called this cruel spirit,
the fell and fen his fastness was, the march his haunt.

With the coming of night came Grendel also,
sought the great house and how the Ring-Danes
held their hall when the horn had gone round.
He found in Heorot the force of nobles
slept after supper, sorrow forgotten,
the condition of men. Maddening with rage,
he struck quickly, creature of evil:
grim and greedy, he grasped on their pallets
thirty warriors, and away he was out of there,
thrilled with his catch: he carried off homeward
his glut of slaughter, sought his own halls.

As the day broke, with the dawn’s light
Grendel’s outrage was openly to be seen:
night’s table-laughter turned to morning’s
lamentation. Lord Hrothgar
sat silent then, the strong man mourned,
glorious king, he grieved for his thegns
as they read the traces of a terrible foe,
a cursed fiend. That was too cruel a feud,
too long, too hard!

Nor did he let them rest
but the next night brought new horrors,
did more murder, manslaughter and outrage
and shrank not from it: he was too set on these things.

So Grendel became ruler; against right he fought,
one against all. Empty then stood
the best of houses, and for no brief space;
for twelve long winters torment sat
on the Friend of the Scyldings, fierce sorrows
and woes of every kind; which was not hidden
from the sons of men, but was made known
in grieving songs, how Grendel warred
long on Hrothgar, the harms he did him
through wretched years of wrong, outrage
and persecution.

A great grief was it for the Guardian of the Scyldings,
crushing to his spirit. The council lords
sat there daily to devise some plan,
what might be best for brave-hearted
Danes to contrive against these terror-raids.

This was heard of at his home by one of Hygelac’s followers,
a good man among the Geats, Grendel’s raidings;
he was for main strength of all men foremost
that trod the earth at that time of day;
build and blood matched.

He bade a seaworthy
wave-cutter be fitted out for him; the warrior king
he would seek, he said, over swan’s riding,
that lord of great name, needing men.

The prince had already picked his men
from the folk’s flower, the fiercest among them
that might be found. With fourteen men
he sought sound-wood; sea-wise \textit{Beowulf}
led them right down to the land’s edge.

Away she went over a wavy ocean,
boat like a bird, breaking seas,
wind-whetted, white-throated,
till the curved prow had ploughed so far
the sun standing right on the second day –
that they might see land loom on the skyline,
then the shimmer of cliffs, sheer fells behind,
reaching capes.

The crossing was at an end;
closed the wake. Weather-Geats
stood on strand, stepped briskly up;
a rope going ashore, ring-mail clashed,
battle-girdings. God they thanked
for the smooth going over the salt trails.

There was stone paving on the path that brought
the war-band on its way. The war-coats shone
and the links of hard hand-locked iron
sang in their harness as they stepped along
in their gear of grim aspect to the hall.

\centerline{\textbf{Scene 2}}
\centerline{\textit{Loud knocking on the door, which then swings open}}
\centerline{\textit{Beowulf \& Companions enter. Wulfgar stands to greet them.}}


}
\end{document}
