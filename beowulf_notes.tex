\documentclass[a4paper]{article}

% Table of contents depth (currently unused)
\setcounter{tocdepth}{3}

% Section numbering depth (zero for no numbering)
\setcounter{secnumdepth}{0}

% latex package inclusions here
%\usepackage{fullpage}
\usepackage{hyperref}
\usepackage{tabulary}
%\usepackage{amsthm}

% set up BNF generator
%\usepackage{syntax}
%\setlength{\grammarparsep}{10pt plus 1pt minus 1pt}
%\setlength{\grammarindent}{10em} 

% set up source code inclusion
\usepackage{listings}
\lstset{
  tabsize=2,
  basicstyle = \ttfamily\small,
  columns=fullflexible
}
% Usage for the above like so:
% \begin{lstlisting}
%   CODE CODE CODE
% \end{lstlisting}

% in-line code styling (same style as listing)
\newcommand{\shell}[1]{\lstinline{#1}}

%%%%%%%%%%%%%%%%%%%%%%%%%%%%%%%%%%%%%%%%%%%%%%%%%%%%%%%%%%%%%%%%%%%%%%%%%%%%%%%

\begin{document}
\title{Beowulf Notes}
\date{2016}
\author{
Daniel Clay \\ 
}
\maketitle

%%%%%%%%%%%%%%%%%%%%%%%%%%%%%%%%%%%%%%%%%%%%%%%%%%%%%%%%%%%%%%%%%%%%%%%%%%%%%%%
\section{Plot}
%%%%%%%%%%%%%%%%%%%%%%%%%%%%%%%%%%%%%%%%%%%%%%%%%%%%%%%%%%%%%%%%%%%%%%%%%%%%%%%

\subsection{Introduction \& Omissions}%%%%%%%%%%%%%%%%%%%%%%%%%%%%%%%%%%%%%%%%%

\textit{Beowulf} is an Anglo-Saxon epic poem, probably written at some point 
between the 6th and 8th centuries and set in 6th century Denmark. The original
is some 3000 lines long, and is written in a primarily West Saxon (IE from Wessex)
dialect of Old English.

To adapt such a poem for the stage requires a good deal of changes to be made
to the script, changes which are detailed in the accompanying document
'Beowulf - Additions and Omissions'. This document aims to provide a set of
'director's notes' to accompany the script, linking the actions in the script
with the passages describing them and providing hints or explanations of details.

\subsection{Notation}%%%%%%%%%%%%%%%%%%%%%%%%%%%%%%%%%%%%%%%%%%%%%%%%%%%%%%%%%%

Parts of the play are referenced by their scene, while lines from the poem are
numbered from the Michael Alexander translation rather than the original.

\subsection{Act 1: Grendel}%%%%%%%%%%%%%%%%%%%%%%%%%%%%%%%%%%%%%%%%%%%%%%%%%%%%

The poem opens with a story about Scyld Shefing, founder of the royal house
of the Spear-Danes, of which \textit{Hrothgar} is a member. \textit{The Bard} 
goes through much of the back story of the Syclding dynasty, the building of Heorot
and introduces Grendel and Beowulf before the play proper begins.

The introduction is taken almost verbatim from the poem, merely missing out sections
for brevity; such as the passage describing Scyld's burial.

While most of the contemporary audience would have been familiar with the story
described (hence the Bard's use of the phrase 'you have heard'), it serves as
a useful introduction to the culture for a modern audience. 

The play itself opens with \textit{Beowulf's} arrival at Heorot (his departure
from Geatland and subsequent arrival in Denmark having been narrated, while his
conversation with the lookout has been omitted). 



\subsection{Act 2: Grendel's Mother}%%%%%%%%%%%%%%%%%%%%%%%%%%%%%%%%%%%%%%%%%%%

Beowulf kills Grendel's Mother.

\subsection{Act 3: The Dragon}%%%%%%%%%%%%%%%%%%%%%%%%%%%%%%%%%%%%%%%%%%%%%%%%%

Beowulf kills the dragon, but in the process is himself killed.


\end{document}
