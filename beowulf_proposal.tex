\documentclass[a4paper]{article}

% Table of contents depth (currently unused)
\setcounter{tocdepth}{3}

% Section numbering depth (zero for no numbering)
\setcounter{secnumdepth}{0}

% latex package inclusions here
%\usepackage{fullpage}
\usepackage{hyperref}
\usepackage{tabulary}
%\usepackage{amsthm}

% set up BNF generator
%\usepackage{syntax}
%\setlength{\grammarparsep}{10pt plus 1pt minus 1pt}
%\setlength{\grammarindent}{10em} 

% set up source code inclusion
\usepackage{listings}
\lstset{
  tabsize=2,
  basicstyle = \ttfamily\small,
  columns=fullflexible
}
% Usage for the above like so:
% \begin{lstlisting}
%   CODE CODE CODE
% \end{lstlisting}

% in-line code styling (same style as listing)
\newcommand{\shell}[1]{\lstinline{#1}}

% Line and paragraph spacing
\newenvironment{linewise}
  {\parindent=0pt
   \obeyspaces\obeylines
   \begingroup\lccode`~=`\^^M
   \lowercase{\endgroup\def~}{\par\leavevmode}}
  {\ignorespacesafterend}

%%%%%%%%%%%%%%%%%%%%%%%%%%%%%%%%%%%%%%%%%%%%%%%%%%%%%%%%%%%%%%%%%%%%%%%%%%%%%%%

\begin{document}
\title{Beowulf Proposal}
\date{23/6/2016}
\author{
Daniel Clay \\ 
}
\maketitle

%%%%%%%%%%%%%%%%%%%%%%%%%%%%%%%%%%%%%%%%%%%%%%%%%%%%%%%%%%%%%%%%%%%%%%%%%%%%%%%
\section{Proposal}
%%%%%%%%%%%%%%%%%%%%%%%%%%%%%%%%%%%%%%%%%%%%%%%%%%%%%%%%%%%%%%%%%%%%%%%%%%%%%%%

\subsection{Introduction}

\textit{Beowulf} is an ancient Anglo-Saxon epic poem, written down somewhere
between the 8th and 11th centuries, but likely composed at some time in the 
6th or 7th centuries. 

Originally, it would have been perfomed in large wooden drinking halls known as 
\textit{Mead Halls} - anyone who has played Skyrim will be familiar with the concept
from Dragonsreach. Likewise, the Golden Hall at Edoras, from The Lord of the Rings is 
a typical example. 

My central idea is to take the idea of the poem declaimed in the mead hall, but instead 
of simply reading the poem, to act it out as well; with actors speaking (and acting) all
the lines of their characters, and the bard narrating sections. 

This, along with the set design (see below) should give an immersive experience
that hopefully makes the audience feel as if they are sat in a mead hall watching the events
unfold before them.

\subsection{Synopsis \& Structure}%%%%%%%%%%%%%%%%%%%%%%%%%%%%%%%%%%%%%%%%%%%%%

\textit{Beowulf} is a hero of the Geatish people (from Scandinavia). On hearing that the
king of the Geat's allies, \textit{Hrothgar} is beset by an evil monster named 
\textit{Grendel}, he travels to Denmark to slay it.

Once in Hrothgar's mead hall \textit{Heorot}, he fights Grendel and kills him.
The rejoicing however is cut short when Grendel's mother attempts to take her 
revenge on the Danes; and Beowulf is forced to travel to her lair and kill her too.

Beowulf then returns to his people covered in glory; and when the Geatish king dies 
he becomes the new king. His reign is long and prosperous, but comes to an end when
a dragon, disturbed from it's hoard attacks. Beowulf, with the help of a companion, 
slays the dragon, but is himself killed. The play ends with a dirge for the fallen hero.
\newline

Despite the length of the original poem, which runs to over 3000 lines, the play
may be able to be performed straight through, without an interval.

If an interval is desired, it should happen after Act II.

The poem itself is loosely organised in three acts, characterised by the three 
combats Beowulf partakes in; that with Grendel, with Grendel's Mother, and with the Dragon.

It should take about 1.5-2.5 hours to perform, exclusive of interval. The truth is,
I don't know how long it will take because the script isn't finished yet. 
Unabridged audiobooks tend to be in the 2-2.5 hour range, and I have cut lots of 
things that are irrelevant to the main plot so that should help keep the length down.

\newpage

\subsection{Cast}%%%%%%%%%%%%%%%%%%%%%%%%%%%%%%%%%%%%%%%%%%%%%%%%%%%%%%%%%%%%%%

In order of appearance:
\begin{description}
    \item[The Bard] Narrator - Large role
    \item[Hrothgar] King of the Spear-Danes (Scyldings)* - Medium role
    \item[Grendel] An evil monster, descendant of Cain* - Medium role
    \item[Beowulf] A hero of the Geatish people - Large role
    \item[Beowulf's companions] 14 Geatish thegns* - Non speaking roles
    \item[Lookout] A warrior of the Spear-Danes who keeps watch for ships* - Small role
    \item[Wulfgar] Hrothgar's Counsellor* - Small role
    \item[Unferth] Hrothgar's spokesman* - Small role
    \item[Wealhtheow] Queen of the Spear-Danes* - Small role
    \item[Hrethic \& Hrothmund] Sons of Hrothgar* - Small roles
    \item[Grendel's Mother] A monstrous ogress* - Medium roles
    \item[Ashhere] Hrothgar's Counsellor* - Small role
    \item[Hygelac] King of the Geats** - Small role
    \item[Hygd] Queen of the Geats** - Small role
    \item[The Dragon] A dragon, awakened from it's slumber** - Medium role (for two people)
    \item[A Slave] Who awakens the dragon** - Small role
    \item[Wiglaf] Beowulf's faithful companion** - Small role
    \item[Beowulf's companions] 7(?) Geatish thegns** - Non speaking roles
\end{description}

Of the cast, most of Beowulf's companions can be omitted as required, and those
present at the court of Hrothgar (marked with a  *) can be doubled up with those present at the court
of Hygelac (marked with a **; characters who cannot be doubled up are unmarked). Hrethic \& Hrothmund may also be omitted.

Likewise, additional \'extras\' may be present at both courts - councillors, warriors and servants.

Most of the minor roles can be played by men or women, as can Grendel, Grendel's
Mother, the Dragon (possibly a two person costume) and the Bard.

In total, the play can be performed by anywhere upwards of 10 people, with an ideal number being around 14-20.

\newpage

\subsection{Creative vision}%%%%%%%%%%%%%%%%%%%%%%%%%%%%%%%%%%%%%%%%%%%%%%%%%%%%

This play works best when the character of the mead-hall, for which the
original work was composed, is retained. As such, the aim would be to make the UCH
resemble as closely as possible a mead hall. 

Staging should be minimal and as far as possible abstracted away. No set changes 
should be required, and scene changes can be suggested by lighting. 

Costumes and properties should be as realistic as possible. Both of these will
require the person responsible to make things rather than buy them, but for the most
part these will not be too difficult to make.

Lighting by open flame, or at least the appearance of open flame, is preferred 
to standard theatre lighting.

Sound would consist mostly of subtle soundscaping to add to the ambience, plus some
sound effects at specific points.

The most notable technical requirement is for three monster costumes - those of
Grendel, Grendel's Mother and the Dragon. These are not something that we make 
often, but we have made similar items before and have the skills to do so again.

Technically, there are lots of things we don't normally do, but nothing that is 
outside of our skill set.

\subsection{Production team}%%%%%%%%%%%%%%%%%%%%%%%%%%%%%%%%%%%%%%%%%%%%%%%%%%%

I will produce, along with an assistant producer.

I'll need a director and AD.

I'll also need lights, sound, costumes, props, set, FX and a stage manager.

Also actors are apparently necessary(?)

\subsection{Rights}%%%%%%%%%%%%%%%%%%%%%%%%%%%%%%%%%%%%%%%%%%%%%%%%%%%%%%%%%%%%

As the oldest surviving work in the English language, \textit{Beowulf} is very
much in the public domain. The translation on which I am basing my script 
(Micheal Alexander's verse translation) isn't however, and I would need to 
negotiate rights with him.

\newpage

%%%%%%%%%%%%%%%%%%%%%%%%%%%%%%%%%%%%%%%%%%%%%%%%%%%%%%%%%%%%%%%%%%%%%%%%%%%%%%%
\section{Script Excerpts}
%%%%%%%%%%%%%%%%%%%%%%%%%%%%%%%%%%%%%%%%%%%%%%%%%%%%%%%%%%%%%%%%%%%%%%%%%%%%%%%

{\linewise

\centerline{\textbf{Act 1, Scene 1}}
\centerline{\textit{A mead hall, somewhere in 6th Century Denmark}}

\textbf{The Bard} Listen!
We have heard of the thriving of the throne of Denmark,
how the folk-kings flourished in former days,
how those royal athelings earned that glory.

Was it not Scyld Shefing that shook the halls,
took mead-benches, taught encroaching
foes to fear him – who, found in childhood,
lacked clothing? Yet he lived and prospered,
grew in strength and stature under the heavens
until the clans settled in the sea-coasts neighbouring
over the whale-road all must obey him
and give tribute. He was a good king!
...

}

\centerline{\textbf{Act 3, Scene 3}}

\centerline{\textit{Beowulf and Wiglaf are fighting the dragon. Beowulf is already injured.\newline}}

\centerline{\textit{The dragon bites Beowulf's neck, but he stabs it with his seax and kills the beast.\newline}}

\centerline{\textit{Beowulf takes a few steps, and sits down on a ledge. Wiglaf tries to attend to his wounds.}}

{\linewise

\textbf{Beowulf, painfully} I would now wish to give my garments in battle
to my own son, if any such
after-inheritor, an heir of my body,
had been granted to me. I have guarded this people 
for half a century; not a single ruler
of all the nations neighbouring about
has dared to affront me with his friends in war,
or threaten terrors. What the times had in store for me
I awaited in my homeland; I held my own,
sought no secret feud, swore very rarely
a wrongful oath.

In all of these things,
sick with my life’s wound, I may still rejoice:
for when my life shall leave my body
the Ruler of Men may not charge me
with the slaughter of kinsmen.

 Quickly go now,
beloved Wiglaf, and look upon the hoard
under the grey stone, now the serpent lies dead,
sleeps rawly wounded, bereft of his treasure.
Make haste, that I may gaze upon that golden inheritance,
that ancient wealth; that my eyes may behold
the clear skilful jewels: more calmly then may I
on the treasure’s account take my departure
of life and of the lordship I have long held.

}

\centerline{\textit{Wiglaf disappears into the wings, and returns with treasures. Beowulf looks over them.}}

{\linewise

\textbf{Beowulf, painfully} I wish to put in words my thanks
to the King of Glory, the Giver of All,
the Lord of Eternity, for these treasures that I gaze upon,
that I should have been able to acquire for my people
before my death-day an endowment such as this.

My life’s full portion I have paid out now
for this hoard of treasure; you must attend the people’s
needs henceforward; no further may I stay.
Bid men of battle build me a tomb
fair after fire, on the foreland by the sea
that shall stand as a reminder of me to my people,
towering high above Hronesness
so that ocean travellers shall afterwards name it
Beowulf’s barrow, bending in the distance
their masted ships through the mists upon the sea.

}

\centerline{\textit{Beowulf takes off his golden collar, helmet and arm ring and passes them to Wiglaf.}}

{\linewise

\textbf{Beowulf, dying} You are the last man left of our kindred,
the house of the Waymundings! Weird has lured
each of my family to his fated end,
each earl through his valour; I must follow them.

}

\centerline{\textit{He dies. Wiglaf sits silently beside him.}}

\end{document}